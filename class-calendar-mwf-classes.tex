\documentclass[11pt]{article} % 12 point doesn't fit on one page
\usepackage[paperwidth=8.5in,paperheight=11in,margin=0.45in]{geometry} 
\usepackage[USenglish]{babel}
\usepackage{colortbl}
\usepackage[final]{microtype}
\usepackage[nodayofweek,level]{datetime} %obsolete--datetime2 is the replacement
\usepackage{calc}
\usepackage[dvipsnames]{xcolor}
\usepackage{ifthen}

% Use \darkerColor for column headings and \lighterColor for 
% highlighted text; for other colors, see \usepackage[dvipsnames]{xcolor}
\newcommand{\darkerColor}{blue!25}
\newcommand{\lighterColor}{blue!15}

% macros for a.m. and p.m.
\newcommand\AM{a.m.} %https://www.unk.edu/ccr/news-internal-communication/writing-style-guide.php
\newcommand\PM{p.m.}
% no page numbers
\pagestyle{empty}

% Counter for homework
\newcounter{hw}\setcounter{hw}{0}
\newcommand{\hw}{%\
\setcounter{hw}{\value{hw}+1}
\textbf{ HW  \thehw}}

% Counter for exams
\newcounter{ex}\setcounter{ex}{0}
\newcommand{\ex}{%\
\setcounter{ex}{\value{ex}+1}
\textbf{ \cellcolor{\lighterColor} Exam \theex}}

% Counter for weeks
\newcounter{wk}\setcounter{wk}{0}
\newcommand{\wk}{%\
\setcounter{wk}{\value{wk}+1}
\thewk \,\,}

% Format a holiday
\newcommand{\holiday}[1]{\textbf{No class} (#1)}

% see https://www.unk.edu/ccr/marketing-advertising/branding-and-identity-marks/typefaces.php
% see also https://tug.org/FontCatalogue/utopia-fouriermath/
\usepackage{fourier}
\usepackage[T1]{fontenc}

% see https://tex.stackexchange.com/questions/325133/advdate-in-tables-does-not-advance-date-correctly
% see also  https://pastebin.com/raw/7pVPQweM
% see https://tex.stackexchange.com/questions/236193/equivalent-of-newdateformat-in-datetime2
% apparently datetime has been replaced....
\usepackage{advdate}
\newdateformat{mydate}{\THEDAY\ \shortmonthname[\THEMONTH]}
\makeatletter
\renewcommand\AdvanceDate[1][\@ne]{%
  \global\advance\day#1 \FixDate
  \global\advance\day\z@
  \global\advance\month\z@
  \global\advance\year\z@}
\makeatother

%nweek: format the first week of the term--it does advance the date
%mweek  format the a class day with out a date or a horizontal line
%eweek  format the last class day of a week (draws a horizontal line)
\newcommand{\nweek}[3]{\wk & \mydate\today  &  #1 & #2 & #3 \AdvanceDate[2] \\}
\newcommand{\mweek}[3]{ &  & #1 & #2 & #3 \AdvanceDate[2]  \\}
\newcommand{\eweek}[3]{ &  & #1 & #2 & #3 \AdvanceDate[3] \\ \hline}

% Print the heading for the course calendar heading


\newcommand{\tablehead}[5]
{\textbf{\cellcolor{\darkerColor}{#1}}  & 
 \textbf{\cellcolor{\darkerColor}{#2}} & 
 \textbf{\cellcolor{\darkerColor}{#3}} & 
 \textbf{\cellcolor{\darkerColor}{#4}} & 
 \textbf{\cellcolor{\darkerColor}{#5}} \\ \hline \hline}
  
% Set this to a message for a review day
\newcommand{\review}{Exam review or catch up}

% Set this to a message for a final review day
\newcommand{\finalreview}{Final Exam review or catch up}

% Set this to a message about an exam day
\newcommand{\examday}{\textbf{Exam Day} (\mydate\today)}

% Message about final exam--the argument is the class time down to
% half hour; so 8:05 --> 800; 9:10 --> 900, and etc.
\newcommand{\finalExamMWF}[1]{\textbf{Final Exam} \finalMWF{#1}}
\newcommand{\finalExamTTh}[1]{\textbf{Final Exam} \finalTTh{#1}}
% Printed at the top of a course calendar--it also resets the ex, hw
% and wk counters.
\newcommand{\calendarHead}[2]{%
  \setcounter{ex}{0}
  \setcounter{hw}{0}
  \setcounter{wk}{0}
  \large
   \begin{flushleft}
    \textbf{Class Calendar} \\
    #1 #2
  \end{flushleft}
  \normalsize}

% Final exam day for MWF classes--round class time down to
% half hour. I think the em-dash is correct; see
% https://tex.stackexchange.com/questions/3819/dashes-vs-vs 
  \newcommand{\finalMWF}[1]
  {
    \ifthenelse{\equal{#1}{800}}{Monday 8:00 \AM---10:00 \AM}{}
    \ifthenelse{\equal{#1}{900}}{Wednesday 8:00 \AM---10:00 \AM}{}
    \ifthenelse{\equal{#1}{1000}}{Monday 10:30 \AM---12:30 \PM}{}
    \ifthenelse{\equal{#1}{1100}}{Wednesday 10:30 \AM---12:30 \PM}{}
    \ifthenelse{\equal{#1}{1200}}{Monday 1:00 \PM---3:00 \PM}{}
    \ifthenelse{\equal{#1}{100}}{Wednesday  1:00 \PM---3:00 \PM}{}
    \ifthenelse{\equal{#1}{200}}{Monday 3:30 \PM---5:30 \PM}{}
    \ifthenelse{\equal{#1}{300}}{Wednesday 3:30 \PM---5:30 \PM}{}
    \ifthenelse{\equal{#1}{400}}{Tuesday 3:30 \PM---5:30 \PM}{}
  }

  \newcommand{\finalTTh}[1]
  {
    \ifthenelse{\equal{#1}{800}}{Tuesday 8:00 \AM---10:00 \AM}{}
    \ifthenelse{\equal{#1}{930}}{Thursday 8:00 \AM---10:00 \AM}{}
    \ifthenelse{\equal{#1}{1100}}{Tuesday 10:30 \AM---12:30 \PM}{}
    \ifthenelse{\equal{#1}{1230}}{Thursday 10:30 \AM---12:30 \PM}{}
    \ifthenelse{\equal{#1}{200}}{Tuesday 1:00 \PM---3:00 \PM}{}
    \ifthenelse{\equal{#1}{330}}{Thursday 1:00 \PM---3:00 \PM}{}
  }

\begin{document}
 % Set to first day of the term--correct for fall 2023
\SetDate[21/08/2023] 
 \calendarHead{Fall}{2023}

 \begin{center}
 \begin{tabular}  {|l|l|l|l|l|}
\hline
\tablehead{Week}{Week of}{Sections}{Topics}{Assessments}
% Week one
\nweek{1.1}{Sets of Real Numbers and the Cartesian Coordinate Plane}{}
\mweek{1.2}{Relations}{}
\eweek{1.3}{Introduction to Functions}{\hw}
% Week 2
\nweek{1.4}{Function Notation}{}
\mweek{1.5}{Function Arithmetic}{}
\eweek{1.6}{Graphs of Functions}{\hw}
  %Week 3
 \nweek{}{ \holiday{Labor Day}}{} 
\mweek{1.7}{Transformations}{}
 \eweek{2.1}{Linear Functions}{\hw}
 % Week 4
\nweek{2.2}{Absolute Value Functions}{}
\mweek{1.1---2.2}{\review}{}
\eweek{}{\examday}{\ex}
% Week 5
 \nweek{2.3}{Quadratic Functions}{}
 \mweek{2.4}{ Inequalities with Absolute Value and Quadratic Functions}{}
 \eweek{2.4}{Inequalities with Absolute Value and Quadratic Functions}{\hw}
 % Week 6
\nweek{3.1}{Graphs of Polynomials}{}
\mweek{3.2}{The Factor Theorem and the Remainder Theorem}{}
\eweek{3.3}{Real Zeros of Polynomials}{\hw}
%Week 7
\nweek{4.1}{Introduction to Rational Functions}{}
\mweek{4.2}{Graphs of Rational Functions}{}
\eweek{4.3}{Rational Inequalities and Applications}{\hw}
% Week 8
\nweek{4.3}{Rational Inequalities and Applications}{}
\mweek{2.3---4.3}{\review}{}
\eweek{}{\examday}{\ex}
% Week 9
\nweek{}{\holiday{Fall Break}}{}   
\mweek{5.1}{Function Composition}{}
\eweek{5.2}{Inverse Functions}{\hw}
% Week 10
\nweek{6.1}{Introduction to Exponential and Logarithmic Functions}{}
 \mweek{6.1}{Introduction to Exponential and Logarithmic Functions}{}
\eweek{6.2}{Properties of Logarithms}{\hw}
% Week 11
\nweek{6.2}{Properties of Logarithms}{}
 \mweek{6.3}{Exponential Equations and Inequalities}{}
  \eweek{6.4}{Logarithmic Equations and Inequalities}{\hw}
  % Week 12
  \nweek{5.1---6.4}{\review}{}
  \mweek{}{\examday}{\ex}
 \eweek{6.5}{Applications of Exponential and Logarithmic Functions}{}
 % Week 13
\nweek{6.5}{Applications of Exponential and Logarithmic Functions}{}
\mweek{8.1}{Systems of Linear Equations: Gaussian Elimination}{}
\eweek{8.1}{Systems of Linear Equations: Gaussian Elimination}{\hw}
  % Week 14
  \nweek{8.2}{Systems of Linear Equations: Augmented Matrices}{\hw}
  \mweek{}{\holiday{Thanksgiving}}{}
  \eweek{}{\holiday{Thanksgiving}}{}
  % Week 15
\nweek{9.1}{Sequences}{}
\mweek{6.5---9.1}{\review}{}
 \eweek{}{\examday}{\ex}
 % Week 16
\nweek{9.2}{Summation Notation}{}
\mweek{9.2}{Summation Notation}{}
\eweek{1.1---9.2}{\finalreview}{}
\nweek{}{\finalExamMWF{800}}{\textbf{ \cellcolor{\lighterColor} Final Exam} } \hline
\end{tabular}
\end{center}

\end{document}
    
